
\section{Ugeopgave 1-3}
\label{sec:ugeopgave-1-3}

Denne ugeopgave besk�ftigede sig med forst�else og videreudvikling af
en raytracer.

En raytracer er en metode til at beregne, hvordan 3D-scener vil
fremst�. Selve \emph{tracingen} foreg�r ved at der udsendes en str�le
fra �jet, der g�r igennem hver pixel i scenen. Denne str�le spores s�,
indtil den rammer et objekt, og returnerer farven, der er p� det
p�g�ldende sted p� det p�g�ldende objekt. Id�t objektets farve kan
�ndres, alt efter hvilke lyskilder, der peger p� det, hvilke
egenskaber objektet har (reflektionsgrad fx) samt hvilke objekter der
er andre steder i scenen, sendes der fra ``nedslagspunktet'' en ny
str�le ud, der bes�ger andre steder i scenen.

Disse str�leudsendinger kan stoppes ved forskellige kriterier:

\begin{itemize}
\item Hvis str�len rammer en perfekt diffus overflade (da der ellers
  skulle udsendes str�ler i alle retninger og dermed uendeligt mange)
\item Hvis str�len er over en hvis generation (dvs. at der fx fra øjet
  maksimalt udsendes 5 str�ler, for at finde v�rdien af �n pixel)
\item Hvis str�lens energi ligger under en forudbestemt t�rskel for
  minimumsenergi
\end{itemize}

\subsection{Del 1: UML-diagram}
\label{sec:del-1:-uml}

F�lgende er et begr�nset UML-diagram over den udleverede raytracer:


\subsection{Shade-funktionalitet}
\label{sec:shade-funktionalitet}

I raytraceren er de fem f�lgende termer implementeret:

\begin{itemize}
\item Omgivende lys (ambient light)
\item Diffus reflektion (Lambertian shading)
\item Specular highlights (Phong highlights)
\item Specular reflections
\item Transmission
\end{itemize}

De f�rste tre er lokale betragtninger, og kr�ver derfor ikke
rekursion, idet det ikke er n�dvendigt at betragte resten af scenen
for at beregne dem.

I vores implementering, udsendes der ikke ``b�rnestr�ler'' ved perfekt
diffuse overflader og desuden stoppes str�lerne efter 5 generationer.

Bortset fra begr�nsningen p� rekursionsdybden er al relevant kode
foretaget i \texttt{Surface.cpp}.

Alle ``bidrag'' fra de forskellige termer adderes og multipliceres til
sidst med objektets grundfarve i det p�g�ldende punkt.

\paragraph{Omgivende lys}
\label{sec:omgivende-lys}

Dette er implementeret ved at multiplicere scenens omgivende lys med
faktoren for omgivende lys, \texttt{k\_ambient}.

\paragraph{Diffus reflektion}
\label{sec:diffus-reflektion}

Dette er implementeret direkte fra formlen, der er givet i bogen:

\[
k_{diffuse} \cdot (\mathbf{N} \cdot \mathbf{L}) \cdot I
\]

Hvor $\mathbf{N}$ er normalvektoren, $\mathbf{L}$ er lysvektoren og
$I$ er lysintesiteterne i scenen.

\paragraph{Specular (phong) highlights}
\label{sec:spec-phong-highl}

Dette er ligeledes implementeret direkte fra formlen i bogen:

\[
(\mathbf{N} \cdot \mathbf{H})^{n} \cdot k_{highlight} \cdot I
\]

I dette tilf�lde er $\mathbf{H}$ den s�kaldte halvejsvektor.



%%% Local Variables: 
%%% mode: latex
%%% TeX-master: "report_main"
%%% End: 
